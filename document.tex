\documentclass[10pt,twocolumn]{article}

% use the oxycomps style file
\usepackage{oxycomps}

% usage: \fixme[comments describing issue]{text to be fixed}
% define \fixme as not doing anything special
\newcommand{\fixme}[2][]{#2}
% overwrite it so it shows up as red
\renewcommand{\fixme}[2][]{\textcolor{red}{#2}}
% overwrite it again so related text shows as footnotes
%\renewcommand{\fixme}[2][]{\textcolor{red}{#2\footnote{#1}}}

% read references.bib for the bibtex data
\bibliography{references}

% include metadata in the generated pdf file
\pdfinfo{
    /Title (GitHub Tutorial)
    /Author (James Derrod)
}

% set the title and author information
\title{GitHub Tutorial}
\author{James Derrod}
\affiliation{Occidental College}
\email{jderrod@oxy.edu}

\begin{document}

\maketitle

\section{Introduction}

This document serves as a tutorial for using the GitHub version control system through the command line and the GitHub Desktop app.


This tutorial will cover the following concepts:
\begin{itemize}
    \item Repositories:
    \begin{itemize}
        \item Initializing repositories.
        \item Cloning repositories.
        \item Adding to a repositories.
        \item Committing to repositories.
        \item Pushing to repositories.
        \item Pulling from repositories.
        \item Stashes.
    \end{itemize}
    \item Merging:
    \begin{itemize}
        \item Working with Branches
        \item Merge conflicts \& resolving them.
    \end{itemize}    
\end{itemize}

Git is a powerful tool for managing your project's development process. Whether you're a solo developer or part of a team, version control with Git offers numerous benefits, including:
\begin{itemize}
    \item Tracking changes: Git keeps a detailed history of every modification made to your project, allowing you to review past iterations and revert to earlier versions if needed.
    \item Collaboration: Git enables seamless collaboration by providing a centralized repository where team members can share and contribute to the project.
    \item Experimentation: With branching, you can explore new features or experiment with different ideas without affecting the main codebase.
\end{itemize}

By the end of this tutorial you should have a concrete understanding of the fundamentals of managing projects with GitHub.


\section{Understanding Git}
Before we dive into the practical aspects of Git, let's take a moment to understand some key concepts:

\begin{itemize}
    \item \textbf{Repository}: A repository, or repo, is a folder or directory where your project files are stored along with the entire history of changes.
    
    \item \textbf{Commit}: A commit represents a snapshot of your project at a specific point in time. Each commit records the changes you've made, along with a descriptive message explaining the modifications.
    
    \item \textbf{Branch}: A branch is a parallel version of your project, allowing you to work on new features or fixes without affecting the main codebase. Branching enables experimentation and facilitates collaboration by isolating changes.
\end{itemize}

Now that you have a basic understanding of Git, let's explore how to perform common tasks.

\section{Initializing a Repository}

Initializing a repository means creating a new Git repository from an existing directory or project that wasn't under version control before.

\begin{itemize}
    \item Command Line: 
    \begin{itemize}
        \item  Navigate to your project directory and execute 'git init' to initialize a new Git repository.
        \item This creates a '.git' directory in your project, containing all necessary metadata for version control.
    \end{itemize}
    \item GitHub Desktop
    \begin{enumerate}
        \item GitHub Desktop does not directly support initializing a repository within the app. You'll need to initialize the repository using the command line or another Git interface, then add it to GitHub Desktop by going to 'File' - 'Add Local Repository'.
    \end{enumerate}
    
\end{itemize}

Initializing a repository is the foundational step to start tracking your project with Git, turning a directory into a workspace that records changes and history. \hyperlink{https://git-scm.com/docs/git-init}{See Git documentation here}. \cite{GitInitDocumentation}\cite{GitGuidesInit}

\section{Cloning a Repository}

Cloning is the process of creating a local copy of a remote repository. It includes all branches and commits, enabling offline work and exploration of the repository's history.

\begin{itemize}
    \item Command Line: 
    \begin{itemize}
        \item Use 'git clone [url]' to clone a repository. 
        \item You can specify a directory name to clone into with 'git clone [url] [directoryName]'.
    \end{itemize}
    \item GitHub Desktop
    \begin{enumerate}
        \item Open GitHub Desktop and log in to your GitHub account.
        \item Go to 'File' - 'Clone Repository'
        \item Choose the repository you wish to clone from the list or specify the URL.
        \item Choose the local path where you want to clone the repository.
        \item Click 'Clone'
    \end{enumerate}
    
\end{itemize}

Cloning is the first step to contribute to a project, allowing you to work with the repository's files locally. \hyperlink{https://git-scm.com/docs/git-clone}{See Git documentation here}. \cite{GitCloneDocumentation}\cite{GitGuidesClone}

\section{Adding to a Repository}
Adding changes, also known as staging, prepares your changes for a commit by marking modifications to be included in the next commit snapshot. This step is crucial for granular control over what is committed.

\begin{itemize}
    \item Command Line: 
    \begin{itemize}
        \item  Use 'git add [file]' to stage a specific file 
        \item Or use 'git add .' to stage all changes in the directory. This command does not alter the repository until a commit is made.
    \end{itemize}
    \item GitHub Desktop
    \begin{enumerate}
        \item Changes are automatically detected and listed.
        \item To stage changes, select the checkbox next to each file you wish to include in your next commit.
    \end{enumerate}
    
\end{itemize}


Adding changes allows you to curate what goes into a commit, ensuring only desired modifications are included, enabling a neat commit history and facilitating precise project tracking. \hyperlink{https://git-scm.com/docs/git-add}{See Git documentation here}. \cite{GitAddDocumentation}\cite{GitGuidesAdd}

\section{Committing to a Repository}
Committing is the process of saving staged changes to the local repository. Each commit has a message that describes the changes, creating a transparent history of your work.
Commits capture the evolution of your project, providing a detailed history of changes. Each commit represents a milestone in your development journey.

\begin{itemize}
    \item Command Line: 
    \begin{itemize}
        \item  Execute 'git commit -m "Commit message"' to commit staged changes with a descriptive message.
        \item For multi-line messages, use 'git commit' without '-m' and an editor will open. 
        \item This editor is determined by your Git configuration, often defaulting to Vim or your system's default text editor. Here, you can write a multi-line commit message, providing the opportunity to include a detailed description of the changes made, why they were made, and any other relevant information.
    \end{itemize}
    \item GitHub Desktop
    \begin{enumerate}
        \item After staging changes, enter a commit message in the summary field.
        \item Click "Commit to [branch]" to save your changes locally
    \end{enumerate}
    
\end{itemize}

Committing encapsulates your progress, allowing you to document each step of development clearly and methodically. \hyperlink{https://git-scm.com/docs/git-commit}{See Git documentation here}. \cite{GitCommitDocumentation}\cite{GitGuidesCommit}

\section{Pushing to a Repository}
Pushing is the act of sending your committed changes to a remote repository. This updates the remote repository with your local commits, sharing your work with the team.

\begin{itemize}
    \item Command Line: 
    \begin{itemize}
        \item  Execute 'git push' to upload local branch commits to the remote repository. 
        \item If you're pushing a new branch, use 'git push -u origin [branchName]' to set the upstream branch.
    \end{itemize}
    \item GitHub Desktop
    \begin{enumerate}
        \item Commit your changes locally.
        \item Click "Push origin" to send your commits to the remote repository.
    \end{enumerate}
    
\end{itemize}

Pushing is essential for collaborative development, allowing others to see and work with your contributions. \hyperlink{https://git-scm.com/docs/git-push}{See Git documentation here}. \cite{GitPushDocumentation}\cite{GitGuidesPush}

\section{Pulling from a Repository}

Pulling updates your local branch with changes from its remote counterpart, merging any new commits from the remote repository into your local working directory.

\begin{itemize}
    \item Command Line: 
    \begin{itemize}
        \item  Use 'git pull' to fetch changes from the remote and merge them into your current branch. This command combines 'git fetch' followed by 'git merge' 
    \end{itemize}
    \item GitHub Desktop
    \begin{enumerate}
        \item Navigate to 'Repository' in the menu bar.
        \item Click 'Pull' to fetch and merge remote changes into your local branch.
    \end{enumerate}
    
\end{itemize}

Pulling ensures your repository stays up-to-date with others' work, facilitating collaboration and minimizing merge conflicts. \hyperlink{https://git-scm.com/docs/git-pull}{See Git documentation here}. \cite{GitPullDocumentation}\cite{GitGuidesPull}

\section{Stashes}

Stashing temporarily shelves (or stashes) changes so you can work on a different task. Your changes are saved on a stack, and you can reapply them later.

\begin{itemize}
    \item Command Line: 
    \begin{itemize}
        \item To stash changes: 'git stash' saves your working directory and index state.
        \item To apply stashed changes: 'git stash pop' re-applies the most recently stashed changes and removes them from the stash stack.
    \end{itemize}
    \item GitHub Desktop
    \begin{itemize}
        \item GitHub Desktop does not directly support stashing through the GUI. Stash management needs to be done via the command line or other Git tools.
    \end{itemize}
    
\end{itemize}

Stashes are useful for switching contexts quickly without committing incomplete work. \hyperlink{https://git-scm.com/docs/git-stash}{See Git documentation here}. \cite{GitPullDocumentation}

\section{Working with Branches}

Branches in Git enable you to diverge from the main line of development and work independently without affecting the main line. Branches enable parallel development, empowering teams to collaborate effectively and experiment with different ideas.

\begin{itemize}
    \item Command Line: 
    \begin{itemize}
        \item To create a new branch: 'git branch [branchName]'
        \item To switch to a branch: 'git checkout [branchName]'
        \item 'git merge [branchName]' - executed from the branch you want to merge into.
    \end{itemize}
    \item GitHub Desktop
    \begin{enumerate}
        \item Click the current branch name in the top menu to open the branch menu.
        \item Click "New Branch" to create and switch to a new branch.
        \item To merge, first switch to the branch you want to merge into. Then go to Branch in the menu, and select 'Merge into current branch'.
    \end{enumerate}
\end{itemize}
Branches are essential for managing features, fixes, and experiments in a segregated manner without disrupting the main codebase. \hyperlink{https://git-scm.com/docs/git-branch}{See Git documentation here}. \cite{GitBranchDocumentation}\cite{GitGuidesRemote}


\section{Merge Conflicts and Resolving Them}

Merge conflicts occur when Git can't automatically reconcile differences in code between two commits. They often happen when two branches have made edits to the same line in a file, or when one branch deletes a file while the other branch was modifying it.
\begin{itemize}
    \item Command Line: 
    \begin{enumerate}
        \item Use 'git status' to identify conflicted files.
        \item Open conflicted files and look for conflicts indicated by ASCII markers in the text editor.
        \item Manually edit the file to resolve conflicts.
        \item After resolving conflicts, use 'git add [file]' to mark as resolved and commit the changes.
    \end{enumerate}
    \item GitHub Desktop
    \begin{itemize}
        \item GitHub Desktop notifies you of conflicts during a merge but does not provide tools to resolve them within the app. You'll need to resolve conflicts in a text editor or IDE, then mark them as resolved through the app.
    \end{itemize}
    
\end{itemize}

Resolving merge conflicts is crucial for maintaining a clean and efficient development process, ensuring that all changes are integrated correctly.

\printbibliography
\end{document}
